%!TEX root = ../template.tex
%%%%%%%%%%%%%%%%%%%%%%%%%%%%%%%%%%%%%%%%%%%%%%%%%%%%%%%%%%%%%%%%%%%%
%% 4_conclusion.tex
%% Rui V. Almeida's thesis file
%%
%% This chapter contains the conclusion presentation.
%%%%%%%%%%%%%%%%%%%%%%%%%%%%%%%%%%%%%%%%%%%%%%%%%%%%%%%%%%%%%%%%%%%%
\chapter{Concluding Remarks and Further Developments}
\label{cha:conclusions}

This dissertation proposes and describes a new atmospheric monitoring
system, based on the use of an autonomous drone, \gls{DOAS} for trace
gas concentration retrieval and tomographic processing for the rendering
of a two dimensional map of the trace gases in a specific geographic
region.

In its beginning,  the project (or rather, its development) was
presented in a Portugal2020 funding round, having been granted the funds
that would allow the actual construction of the system. In the end this
would be the case due to some bureaucratic issues rising from the need
to change the project's promoting company.

With the system's physical construction out of the way, the work evolved
into one of gathering sufficient evidence for a simulation-based proof
of concept. Replying to the Research Questions of
Section~\ref{sec:research_questions} became a matter of testing the two
hypothesis described in the introductory text of
Chapter~\ref{cha:methods}. It was thus that I divided this concluding
chapter, with Section~\ref{sec:tomosim_conclusions} addressing the data
assembly and treatment as a tomographic problem (which has since been
published, see Appendix~\ref{ap:tomosim_paper}) and
Section~\ref{sec:lambertian_absorption_experiment} addressing the second
hypothesis, described in Section~\ref{sec:second_hypothesis}. 

A very short summary table aiming to provide a brief reply to the
aforementioned research questions is provided in
Table~\ref{tab:rq_summary}.

\begin{table}[htb]
    \centering
    \caption{Summary table: this table aims to provide a brief and
    concise reply to the research questions identified in
    Section~\ref{sec:research_questions}.}
    \label{tab:rq_summary}
    \begin{tabularx}{\textwidth}{l l X}
        \toprule
        Research Question & Status  & Description\\
        \midrule

        RQ 1.1 & \cellcolor[HTML]{FFEB9C}\color[HTML]{AA6A17}Partially
        Verified& \cellcolor[HTML]{FFEB9C}\color[HTML]{AA6A17}The
        Lambertian hypothesis of Section~\ref{sec:second_hypothesis}
        would allow for the coverage of the target geographical area
        with an autonomous \gls{UAV}.  This hypothesis was only
        partially verified and needs further studies with more
        sophisticated equipment. \\

        \midrule

        RQ 1.2 & \cellcolor[HTML]{C6EFCD}\color[HTML]{367E33}Verified /
        Published &\cellcolor[HTML]{C6EFCD}\color[HTML]{367E33}Described in
        Appendix~\ref{ap:tomosim_paper}. \\

        \midrule

        RQ 1.3 & \cellcolor[HTML]{C6EFCD}\color[HTML]{367E33}Verified /
        Published &\cellcolor[HTML]{C6EFCD}\color[HTML]{367E33}By performing a
        series of spectroscopic measurements with a particularly defined
        trajectory, it will be possible to gather fanbeam projection
        data for the target geographical region. \\

        \midrule

        RQ 1.4 & \cellcolor[HTML]{C6EFCD}\color[HTML]{367E33}Verified /
        Published &\cellcolor[HTML]{C6EFCD}\color[HTML]{367E33}Described in
        Section~\ref{sec:tomosim}\\

        \bottomrule
        
    \end{tabularx}
\end{table}




\section{TomoSim}%
\label{sec:tomosim_conclusions}

The initial goal of the TomoSim software project was to develop a
simulation platform with which to recreate the tomographic
reconstruction of the column density distribution for a number of target
atmoshperic trace gases. The software program was written using the
Python language and some numeric calculation libraries, such as NumPy
and SciPy. Using these two libraries had two main effects: on the one
hand, it enabled the programmers to easily create and manipulate
matrices and vectors (images, for instance), and on the other, they
greatly improved the running speed of the code, since their core is
written in lower level languages (namely C).

TomoSim runs three algorithms on the projection data in order to produce
the spectral mapping of the target pollutants: FBP (analytical), SART and
MLEM (both algebraic). SART offered the best results, at the expense of
time. The analytical algorithm produced very nearly the same results,
but took a fraction of the time when comparing with either SART of MLEM.
The MLEM algorithm cannot be directly compared to the SART algorithm,
due to differences in the optimisation levels of both routines, but had
nonetheless a reasonable time performance altogether, although producing
the poorest reconstruction results.

The simulations that the software performs prove that, if the final
device is programmed to comply to trajectory and acquisition
requirements, reconstruction is perfectly achievable, even with
relatively low projection numbers (comparing with medical imaging
procedures). This brings another significant conclusion which is that
the devised acquisition definitions, which produce a set of fan beam
arrays, provide sufficient projection information to run the
reconstruction and achieve plausible results.


Regarding future developments, there are three main avenues that should
be explored: 
\begin{description}

    \item[Other phantoms:] Presently, TomoSim only includes tomographic
        reconstruction for two different phantoms.  While this is
        sufficient for simulation, it would be desirable to have some
        more phantoms, which could mimic other concentration
        distributions of interest.

    \item[Paradigm shift:] This simulation software was developed under
        the passive DOAS analysis model. Active measurements are much
        more versatile and accurate, and it would be interesting to
        develop this same technique using an artificial light source. Of
        course this would require many adaptations, namely regarding
        equipment and trajectory (probably even algorithms and
        interpolations).

    \item[Threedimensional reconstruction:] TomoSim was developed to
        produce the reconstruction of a twodimensional image
        corresponding to the spatial distribution of an array of target
        trace gases. It would be much more interesting to have a
        threedimensional equivalent. As far as simulation goes, this is
        one of the most immediate developments for this project. On a
        more tangible level, the additional dimensional would make the
        problem much more complex, mainly because of trajectory and
        battery logistics.
\end{description}

\section{Lambertian Absorption Experiment}%
\label{sec:lambertian_absorption_experiment}

The experiment described in Section~\ref{sec:the_experiment} and
discussed in Section~\ref{sec:second_hypothesis} was designed to verify
if one could, with currently commercially available equipment, perform
\gls{DOAS} measurements for small distances, as an autonomous \gls{UAV}
would perform. If this hypothesis was confirmed, the atmospheric
monitoring system that I proposed in this thesis would be pretty much
completed. Although this is not directly the case, I have reasons to be
optimistic. Namely that the density values obtained for both the active
and the passive \gls{DOAS} measurements were in the same order of
magnitude and displayed the same trends. One would not be realistic if
expecting the measurements to be overlappable, and the fact that they
seem to have similar behaviour is very positive. Moreover, I have
identified an important imperfection in this experiment, which is the
fact that the alignment can be very complicated to ensure in a less than
perfect day. The two together make it impossible to produce definite
conclusions without further study, namely with an automatically aligned
system, but they also leave some good room to consider my hypothesis
very plausible. This is highlighted even more when considering that the
comparison with the official public data returns two very different
results for the two runs of the experiment: on the first run (when the
weather was perfect and there was no wind at all), my results are
similar to the ones obtained from QualAR; on the second run (when the
day was quite windy), they do not agree at all.

With regards to future developments for this side of the project, I
think it is imperative to develop the necessary adaptations for this
system to operate autonomously, so that we can solve the question of
alignment once and for all.
