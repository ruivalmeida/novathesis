%!TEX root = ../template.tex
%%%%%%%%%%%%%%%%%%%%%%%%%%%%%%%%%%%%%%%%%%%%%%%%%%%%%%%%%%%%%%%%%%%%
%% abstrac-pt.tex
%% NOVA thesis document file
%%
%% Abstract in Portuguese
%%%%%%%%%%%%%%%%%%%%%%%%%%%%%%%%%%%%%%%%%%%%%%%%%%%%%%%%%%%%%%%%%%%%

\typeout{NT FILE abstrac-pt.tex}

Era o objectivo deste trabalho descrever o processo de desenho e
implementação de uma prova de conceito para um sistema de avaliação
atmosférica comercialmente viável, para uso no mapeamento das
concentrações de compostos traço na atmosfera.

A avaliação atmosférica é um campo muito estudado, estando no presente
momento disponíveis para instalação diversos sistemas e subsistemas com
estas capacidades. No entanto, é marcante o compromisso que se verifica
entre a resolução espacial e a complexidade operacional destes
equipamentos. Nesta tese, desafio este problema e levanto a questão
sobre como se poderia desenvolver um sistema com os mesmos fins, mas sem
este premente compromisso. 

Usando uma metodologia a duas partes, proponho duas hipóteses  para
comprovar a exequibilidade deste sistema. A primeira diz respeito à
formação da matriz tomográfica e à resolução das equações que dela
derivam e que formam a imagem que se pretende. Confirmei esta hipótese
teoricamente através do desenvolvimento de uma plataforma de simulação
para a reconstrução tomográfica de um campo de concentrações fantoma.

A segunda é dirigida a aquisição de dados espectroscópicos. Proponho que
com o material presentemente disponível comercialmente, deverá ser
possível aproveitar uma consequência da lei de Beer-Lambert para retirar
os valores de concentração molecular de gases traço na atmosfera. Foi
apenas possível validar esta hipótese parcialmente, sendo que resultados
mais conclusivos necessitariam de equipamentos automatizados dos quais
não foi possível dispôr.

No final, este trabalho constitui um importante primeiro passo no
estabelecimento da técnica de \gls{DOAS} tomográfico como uma
alternativa comercialmente viável para a análise atmosférica. Ademais, o
desenvolvimento desta tese levou à escrita de uma biblioteca em Python
para análise de dados \gls{DOAS} actualmente usada em ambiente de
produção. Por fim, importa realçar que dos trabalhos realizados no
decorrer da tese foram publicados dois artigos em revistas científicas
com \emph{Impact Factor} acima de 3.

% Palavras-chave do resumo em Português
\begin{keywords}
    \gls{DOAS}, tomografia, drones
\end{keywords}
% to add an extra black line
