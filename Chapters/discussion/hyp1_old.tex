%!TEX root=../../template.tex
\section{First Hypothesis}%
\label{sec:first_hypothesis}

Results presented in Section~\ref{sec:tomosim}  raise a series of
pertinent observations that should be addressed in discussion. The first
remark goes to the fact that the number of projections used seemed
adequate to perform the tomographic reconstruction in adequate fashion,
although there are tomographic applications with many more projections
(orders of magnitude more projections).  This was not an unexpected
result, as there are already several examples of studies in which a
considerably lower number of projections was used, still producing
satisfactory results~\cite{Pundt2005}.  Moreover, and since the
simulation software automatically includes errors in the calculations,
this number also proves that geometric error plays a very limited role
in changing the result of the reconstruction due to the difference in
size between the drone's trajectory and the geometric error, this was
also a predicted result.

Even with the relatively low number of projections produced by the
drone's trajectory and measurement strategy, all reconstruction
algorithms were able to produce a reconstructed image that resembled the
phantom that generated it. $\Delta$, the angular interval between
projections, revealed itself to be crucial. This was expected as the
number of captured projections is obtained by dividing 360 by $\Delta$.
This is, also as expected, confirmed by the upward trend of the error
when increasing the value of the angular projection interval, as can be
seen numerically in Table~\ref{tab:rec_errors} and qualitatively in
Figure~\ref{fig:rec_comparison}. With respect to the algorithms used,
the custom-made MLEM routine produces clearly outlier results, which are
not on par with the other two reconstruction methods used. This is plain
to see both in Figure~\ref{fig:rec_results} and in
Table~\ref{tab:rec_errors}, in which this algorithm's NRMSE is almost
four times the second best result (FBP) for the smallest projection
interval. This difference could to some extent be expected. SART and FBP
algorithms were implemented using some of the most relevant and
consistently used Python libraries (SciPy, for
instance~\cite{Virtanen2020}). Given the amount of attention these
libraries get from the scientific programming community, levels of
optimisation are extremely high. Although it is nowhere near the other
two approaches, the MLEM routine is still useful, as it is the only
truly geometry-independent algorithm in this study (SART is also
geometry independent, but this particular implementation expects a
parallel projection sinogram as input).

As stated in Section~\ref{sub:error_estimation}, three different kinds
of error influence the reconstruction results: geometric, spectroscopic
and reconstruction errors. The first kind of error is directly included
in projection calculations, through the application of a Monte
Carlo-like method to the geometry described in the same section. The
second kind of uncertainty comes from the spectrum acquisition process
itself, which is not perfect.  If one considers there are no systematic
errors present in the results, which is an acceptable premise in a
simulation, then these errors can be simulated by the inclusion of
Gaussian noise in the spectral measurements. This approach is based on
the one used in~\cite{Stutz1996}, in which a Gaussian noise spectrum is
added to the spectrum of interest in order to simulate how the error
behaves with a degraded signal. Finally, reconstruction errors come from
the finite precision of the calculations that render the images. These
errors were presented in Subsection~\ref{sub:reconstruction_results}.

The three methods were also evaluated as to how they perform
computationally, by measuring the time it took to produce the images in
Figure~\ref{fig:rec_results} using a Paperspace P4000 cloud computing
instance. In this regard, the fastest method was FBP, which took around
3 seconds to reconstruct. The second was MLEM, with around 50 seconds
for 1000 iterations, and finally came SART, with 1 minute and 50 seconds
for 1 iteration. One relevant observation comes from the fact that MLEM
was significantly faster than SART, even taking into account the
difference in optimisation, which was not an expected result and may
indicate some reconstruction enhancing technique on SART's side, as the
literature seems to indicate that this technique is faster than
MLEM~\cite{Defrise2003}.

All things considered, the FBP algorithm produces a very good
reconstruction, equivalent to SART's, while being more than 10 times
faster, indicating that for this kind of application and with this kind
of projection information, it is the best reconstruction algorithm.
