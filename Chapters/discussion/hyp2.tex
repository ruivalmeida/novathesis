%!TEX root=../../thesis_rui_almeida.tex
\section{Second Hypothesis}%
\label{sec:second_hypothesis}

The proposed hypothesis was, as explained in
Section~\ref{sec:intro_hypothesis}, that one could obtain a trace gas'
column density between two points by determining this value on both said
points and then subtracting them one from another. Moreover, this was
achievable with the kind of distances that the proposed monitoring
system would be designed to use. To test this, I have setup an
experiment comprising two spectroscopic assemblies. With them, I was
able to test my approach against an active \gls{DOAS} system. The
expected results included similar levels of detected molecules for the
passive and the active systems.

I defined the protocol and divided into measurements, which are
comprised of sets of unitary actions. These actions included not only
the measurements themselves, but also the collection of reference data
for each measurement moment. In order to guide the conduction of the
experiment, both on my behalf and on the volunteer's, a written protocol
was created and made available to participants. This protocol
constitutes an important part of Appendix~\ref{ap:protocol}.

The experiment was conducted in two "runs". The first run took place on
07/06/2021. A slight logistics-related delay complicated matters. The
first measurement was taken at 06:20, some minutes after sunrise.
Although meteorological conditions were optimal and there were no
hassles, work was interrupted at around 08:20, when the battery of one
of the laptops that were being used ran out. The collected data on this
first run were not as perfect as they seemed to be. An undetected
operational error (with probable protocol breach) caused some of the
data to be missing, and some of the collected spectra were
"contaminated" - meaning measurement interruptions were not correctly
registered. This alone dictated the need to have another go at the
experiment. Even with the missing data, I ran the processing routines on
the data that existed. Part of the results were as expected, another
part puzzling. The puzzling part was that the highest registered column
density happened on the second measurement, before 07:00. After that,
the number of detected molecules evolves as one would expect: rising
with traffic intensity. The expected (and positive) part was that
density evolution was the same for both the passive and the active
\gls{DOAS} systems.

The second run of the experiment...




% As stated in the introductory paragraphs of Chapter~\ref{cha:methods},
% the proposed hypothesis was:

% \begin{center}
% \end{center}
% \begin{center}
%     \begin{minipage}{0.8\textwidth}

%         \noindent\textit{We can retrieve the column density for a given
%         trace gas (or set of trace gases) between two points by
%         performing a spectral measurement in both of these points in the
%         same direction and subtracting them from one another.}

%     \end{minipage}
% \end{center}
% In Section~\ref{sec:the_experiment}, I have argued that this hypothesis
% is in accordance with Lambert-Beer's law and theory, but that in the
% real world there might be some practical limitations that needed to be
% investigated. To test this hypothesis, I have setup an experiment that
% aimed to compare column densities retrieved by an active \gls{DOAS}
% assembly with those that were taken with a scattered sunlight assembly
% in the hypothesis conditions (two measurements in the same direction).
% Materials and equipment are presented in Table~\ref{tab:assemblies}.
% Experiment results (measured trace gas concentrations) are presented in
% Table~\ref{tab:experiment_results}. 

% \begin{table}[htpb]
%     \centering
%     \caption{Numerical results for the experiment detailed in
%     Section~\ref{sec:the_experiment}. Both active and passive modes of
%     operation evolve in approximately the same way, confirming the initial
%     hypothesis.}
%     \label{tab:experiment_results}
%     \begin{tabular}{c}
%     \missingfigure{}
%     \end{tabular}
% \end{table}

% The optical path for the experiment was above a very busy highway.
% Moreover, the experiment was programmed to run during the time period
% that builds up to the morning rush hour. NO\textsubscript{2}, a
% trace gas that is produced by diesel combustion~\todo{Fetch reference
% for this}, was expected to rise in concentration during the experiment.
% This was precisely what happened in both the active and passive
% experiment configurations. Besides, column density values, i.e., the
% number of molecules of that trace gas that interacted with the light,
% were also similar in both experiment modes. Our system's measurement
% strategy was therefore validated.

% \todo[inline]{This section lacks a bit of polish.}
