%!TEX root=../../thesis_rui_almeida.tex
\section{Second Hypothesis}%
\label{sec:second_hypothesis}

As stated in the introductory paragraphs of Chapter~\ref{cha:methods},
the proposed hypothesis was:

\begin{center}
\end{center}
\begin{center}
    \begin{minipage}{0.8\textwidth}

        \noindent\textit{We can retrieve the column density for a given
        trace gas (or set of trace gases) between two points by
        performing a spectral measurement in both of these points in the
        same direction and subtracting them from one another.}

    \end{minipage}
\end{center}
In Section~\ref{sec:the_experiment}, I have argued that this hypothesis
is in accordance with Lambert-Beer's law and theory, but that in the
real world there might be some practical limitations that needed to be
investigated. To test this hypothesis, I have setup an experiment that
aimed to compare column densities retrieved by an active \gls{DOAS}
assembly with those that were taken with a scattered sunlight assembly
in the hypothesis conditions (two measurements in the same direction).
Materials and equipment are presented in Table~\ref{tab:assemblies}.
Experiment results (measured trace gas concentrations) are presented in
Table~\ref{tab:experiment_results}. 

\begin{table}[htpb]
    \centering
    \caption{Numerical results for the experiment detailed in
    Section~\ref{sec:the_experiment}. Both active and passive modes of
    operation evolve in approximately the same way, confirming the initial
    hypothesis.}
    \label{tab:experiment_results}
    \begin{tabular}{c}
    \missingfigure{}
    \end{tabular}
\end{table}

The optical path for the experiment was above a very busy highway.
Moreover, the experiment was programmed to run during the time period
that builds up to the morning rush hour. NO\textsubscript{2}, a
trace gas that is produced by diesel combustion~\todo{Fetch reference
for this}, was expected to rise in concentration during the experiment.
This was precisely what happened in both the active and passive
experiment configurations. Besides, column density values, i.e., the
number of molecules of that trace gas that interacted with the light,
were also similar in both experiment modes. Our system's measurement
strategy was therefore validated.

\todo[inline]{This section lacks a bit of polish.}
