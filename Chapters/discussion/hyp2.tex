%!TEX root=../../template.tex
\section{Second Hypothesis}%
\label{sec:second_hypothesis}

The proposed hypothesis was, as explained in
Section~\ref{sec:intro_hypothesis}, that one could obtain a trace gas'
column density between two points by determining this value on both said
points and then subtracting them one from another. Moreover, this was
achievable with the kind of distances that the proposed monitoring
system would be designed to use. To test this, I have setup an
experiment comprising two spectroscopic assemblies. With them, I was
able to test my approach against an active \gls{DOAS} system. The
expected results included similar levels of detected molecules for the
passive and the active systems.

I defined the protocol and divided into measurements, which are
comprised of sets of unitary actions. These actions included not only
the measurements themselves, but also the collection of reference data
for each measurement moment. In order to guide the conduction of the
experiment, both on my behalf and on the volunteer's, a written protocol
was created and made available to participants. This protocol
constitutes an important part of Appendix~\ref{ap:protocol}.

The experiment was conducted in two "runs". The first run took place on
07/06/2021. A slight logistics-related delay complicated matters. The
first measurement was taken at 06:20, some minutes after sunrise.
Although meteorological conditions were optimal and there were no
hassles, work was interrupted at around 08:20, when the battery of one
of the laptops that were being used ran out. The collected data on this
first run were not as perfect as they seemed to be. An undetected
operational error (with probable protocol breach) caused some of the
data to be missing, and some of the collected spectra were
"contaminated" - meaning measurement interruptions were not correctly
registered. This alone dictated the need to have another go at the
experiment. Even with the missing data, I ran the processing routines on
the data that existed. Part of the results were as expected, another
part puzzling. The puzzling part was that the highest registered column
density happened on the second measurement, before 07:00. After that,
the number of detected molecules evolves as one would expect: rising
with traffic intensity. The expected (and positive) part was that
density evolution was the same for both the passive and the active
\gls{DOAS} systems.

The second run of the experiment happened later in the same month, on
the 23\textsuperscript{rd}. A new challenge came from the fact that the
experiment's site was re-located to a better place (in terms of
measuring capabilities),  the alignment became more difficult, and
revealed itself to be of paramount importance, rendering several
measurements useless. Also, in this second day of experiments, there was
a lot more wind than in the first day, which also played an important
part in this alignment problem. With the 8 viable measurements that were
left, I calculated molecular densities for \gls{no2} and gls{o3}, which
are displayed in Figure~\ref{fig:no2_o3_densities}. The \gls{DOAS} fits
that led to these concentration values are also displayed, in
Figure~\ref{fig:doas_fits}. Official numbers for \gls{no2}
concentrations are provided by QualAR, a Portuguese public initiative
that measures air quality throughout the country~\todo{citation}. The
QualAR chart for the relevant day is presented in
Figure~\ref{fig:qualar_june_23}. 

There are two important levels of analysis for the density plots
presented in Figure~\ref{fig:no2_o3_densities}:
\begin{description}
    \item[Internal analysis:] did the experiment confirm or deny the
        hypothesis it was meant to test (see
        Section~\ref{sec:second_hypothesis})?
    \item[External analysis:] are the collected data comparable in any
        way with the official data coming from QualAR?
\end{description}

With regards to the first angle of analysis, a first glance comparison
of active and passive-related measurement charts might indicate that the
hypothesis was firmly denied. However, one should look at the charts
with some care. Both types of chart are always in the same scale,
meaning that whatever is being measured, it appears to have
approximately the same effect on both measurements. Moreover, if it is
true that they exhibit different behaviours, it is also true that with
respect to a trend, they seem to agree. The analysed data are too
different in order to issue a confirmation of the proposed hypothesis,
but they are also too close to conclude the opposite. Once misalignment
issues are factored in, this becomes ever more evident. The hypothesis
remains plausible, even if not confirmed. 

The second lens through which one must look at the charts is the
external analysis. This must also be done with some care. The first
important thing to consider is that absolute quantities are of no
importance here. The data collected through this experiment was not
processed with the goal of providing a vertical column density that
could be compared to the official numbers. Besides this, although the
air quality station that provided the QualAR data is not far from the
place in which I ran the experiment, it is still sufficiently far that
the absolute values should not be the same.

The QualAR chart tells us that, from 0700 onwards, \gls{no2}
concentration only goes down. This is in direct agreement with the
\gls{no2} cycle described in Section~\ref{sub:criteria_pollutants}.
Although not being as monotonic as the official data, our density charts
(see Figure~\ref{fig:no2_o3_densities}) confirm this. Two factors can be
used to explain the behaviour difference that can be seen between our
charts and QualAR's:
\begin{itemize}
    \item misalignment issues may have introduced this discrepancy as an
        error;
    \item measurements may have been too close to the noise level, and
        therefore some instability is expected.
\end{itemize}

The second hypothesis is clearly the more worrisome, as it threatens the
validity of the whole experiment. However, the fit charts in
Figure~\ref{fig:doas_fits} make this unlikely, since one can clearly see
the signal above the noise.
