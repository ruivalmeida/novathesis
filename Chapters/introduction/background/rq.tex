%!TEX root = ../../../template.tex

\section{Research Questions}%
\label{sec:intro_research_questions}

In Section~\ref{sub:intro_context}, I have introduced the reasons which led
NGNS-IS to pursue the development of an atmospheric monitoring system,
and that what set it apart from other systems was the ability to
spectroscopically map pollutants concentrations using tomographic
methods, thus defining a primary objective for this thesis.

Two secondary objectives were born from the necessary initial research,
which had a very heavy influence over the adopted methods:
\begin{itemize}
    \item To use a tomographic approach for the mapping procedure;
    \item To ensure the designed system would be small and highly
        mobile;
    \item To use a single light collection point, minimizing material
        costs.
\end{itemize}

Taking all the above into account, we arrive at the main Research
Question (\gls{RQ}), presented in Table~\ref{tab:RQ1}.
\begin{table}[htpb]
    \centering
    \caption{Main research question.}
    \label{tab:RQ1}
    \begin{tabularx}{0.8\textwidth}{cX}
        \toprule
        \textbf{RQ1}&\emph{ How to design a miniaturized tomographic
        atmosphere monitoring system based on DOAS? }\\
        \bottomrule
    \end{tabularx}
\end{table}

This is the main research question. It gave rise to four other more
detailed research questions. These secondary questions allow a better
delimitation of the work at hand and are important complements to RQ1.
This questions are presented in Table~\ref{tab:sec_RQ}.

\begin{table}[htpb]
    \centering
    \caption{Secondary research questions.}
    \label{tab:sec_RQ}
    \begin{tabularx}{0.8\textwidth}{cX}
        \toprule
        \textbf{RQ1.1}&\emph{ What would be the best strategy
        for the system to cover a small geographic region? }\\
        \midrule
        \textbf{RQ1.2}&\emph{ What would be the necessary
        components for such a system? }\\
        \midrule
        \textbf{RQ1.3}&\emph{ How will the system acquire the
        data? }\\
        \midrule
        \textbf{RQ1.4}&\emph{ What should the tomographic
        reconstruction look like and how to perform it? }\\
        \bottomrule
    \end{tabularx}
\end{table}
