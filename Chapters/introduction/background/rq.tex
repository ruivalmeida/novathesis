%!TEX root = ../../../template.tex

\section{Research Questions}%
\label{sec:intro_research_questions}

This thesis had the defining primary objective of modelling a proof of
concept for a spectroscopy-based pollution monitoring system that can be
fitted onto a highly mobile platform, such as a drone. The other
defining characteristic of the system is that it needs to be able to
cover large areas (which could be remote and / or inaccessible) and have
a good spatial resolution.

Early in the project's life, the team reached a conceptual milestone:
what if it was possible to perform an atmospheric tomographic scan?
Preliminary research into the literature indicated this had already been
done (focused literature review available in
Chapter~\ref{cha:literature_review}). This started to systematise our
objectives.

\begin{itemize}
    \item To use a tomographic approach for the mapping procedure;
    \item To ensure the designed system would be small and highly
        mobile;
    \item To use a single light collection point, minimizing material
        costs.
\end{itemize}

The main research question was thus formed, and is presented in
Table~\ref{tab:RQ1}. From it, four secondary research questions are
derived, which are presented in Table~\ref{tab:sec_RQ}.

\begin{table}[htpb]
    \centering
    \caption{Main research question.}
    \label{tab:RQ1}
    \begin{tabularx}{0.8\textwidth}{cX}
        \toprule
        \textbf{RQ1}&\emph{ How to design a miniaturized tomographic
        atmosphere monitoring system based on \gls{DOAS}? }\\
        \bottomrule
    \end{tabularx}
\end{table}

\begin{table}[htpb]
    \centering
    \caption{Secondary research questions.}
    \label{tab:sec_RQ}
    \begin{tabularx}{0.8\textwidth}{cX}
        \toprule
        \textbf{RQ1.1}&\emph{ What would be the best strategy
        for the system to cover a small geographic region? }\\
        \midrule
        \textbf{RQ1.2}&\emph{ What would be the necessary
        components for such a system? }\\
        \midrule
        \textbf{RQ1.3}&\emph{ How will the system acquire the
        data? }\\
        \midrule
        \textbf{RQ1.4}&\emph{ What should the tomographic
        reconstruction look like and how to perform it? }\\
        \bottomrule
    \end{tabularx}
\end{table}
