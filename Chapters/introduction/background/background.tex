%!TEX root = ../../../template.tex

\section{Starting Points}%
\label{sec:intro_starting_points}

This thesis describes the work that I have done in the past 4 years on
the design and development of a miniaturized system for atmospheric
monitoring based on optical spectroscopy. The project itself was the
major part of the \gls{ATMOS}, an initiative that was
contemplated with European funding through a \gls{PT2020} initiative and
came as a response to the growing weight that \gls{AP} has in the whole
Western world.

The potential impact of \gls{AP} on human health is amply documented.
Numerous papers have, for decades, established many links between air
quality and several common ailments like respiratory syndromes and
cardiovascular diseases. Similar connections have also been found
regarding the probability of gestational malformations and several types
of cancer. On a different level, and of perhaps less immediate concern,
are the effects that have been observed on ecosystems. Many times these
effects are difficult to predict (and timely mitigate) and in some cases
have been known to interfere with people's livelihood. In time, and if
not addressed, these interferences will certainly hinder economies and
limit the quality of life of populations globally. The severity of this
problem makes it clear that we need to tackle it intelligently, and this
approach requires that we can measure, trace and track \gls{AP}
effectively, which beckons engineers and scientists to create more
technology for this specific purpose.

% Answering this call, with this work I have tried to create a reply to
% the question of whether it would be possible to develop a
% two-dimensional pollutant mapping tool, small enough to be fitted onto a
% \gls{UAV}, which came to be a tomographically enabled design. To this
% end, I have developed a simulation platform that computationally proves
% the method's feasibility and confirmed through experiments that the
% hypothesis on which the solution is based, regarding the use of
% sequentially measured scattered sunlight as analogous to an artificial
% light source is valid.

% \subsection{Context}%
% \label{sub:intro_context}

The idea behind this thesis was born in 2015, at NGNS-IS (a Portuguese
tech startup). At the time, the company's flagship product was the
\gls{FFF}. The \gls{FFF} was a forest fire detection system, capable of
mostly autonomous and automatic operation.  The system was the first
application of \gls{DOAS} for fire detection, and for that it was
patented in 2007 (see~\cite{Vieira2007, Application2008}). The \gls{FFF}
is a remote sensing device that scans the horizon for the presence of a
smoke column, sequentially performing a chemical analysis of each
azimuth, using the Sun as a light source for its spectroscopic
operations~\cite{ValentedeAlmeida2017}.

The \gls{FFF} was deployed in several "habitats", both nationally
(Parque Nacional da Peneda-Gerês and Ourém) and internationally (Spain
and Brazil). One of the company's clients at the time was interested in
a pollution monitoring solution, and asked if the spectroscopic system
would be capable of performing such a task. The challenge resonated
through the company's structure and the idea that created this thesis
was born. The team then started reading about the concept of \gls{AP}
and how both populations and entities were concerned about it. It became
clear that, while there were already several methods to measure
\gls{AP}, there was a clear market drive for the development of a system
that could leverage the large area capabilities of a \gls{DOAS} device
while being able to provide a more spatially resolved "picture" of the
atmospheric status. With this in mind, the company managed to have the
investigation financed through a \gls{PT2020} funding opportunity. This
achievement was a clear validation of the project's goals and of the
need there was for a system with the proposed capabilities. It was,
however, not enough. \gls{FFF} was a very good starting point, but there
was still a lot of continuous research work needed before any of the
goals that had been set were achieved. This led to the publication of
this PhD project, in a tripartite consortium between FCT-NOVA, NGNS-IS
and the Portuguese Foundation for Science and Technology. Its main goal
was to develop an atmospheric monitoring system prototype that would be
able to spectroscopically map pollutant concentrations in a
two-dimensional way.

In April 2017, NGNS-IS was integrated in the Compta group, one of the
oldest IT groups operating in Portugal. Despite its age, this company is
one of the main presences in some of the most modern industrial fields,
like \gls{IOT} applications. \gls{ATMOS}'s pollutant tracing
capabilities made it an almost perfect fit in one of \gls{IOT}'s most
resounding niches, the \emph{Smart Cities} trend. Unfortunately, the
transition between one company and the other, regardless of the
project's adequacy, was anything but smooth. Almost two years later, in
the beginning of 2019, engulfed in a sea of endless bureaucracy and ill
intent on behalf of the managing governmental authorities (who seemed
always more interested in seeing the project fail than anything else),
\gls{ATMOS} was terminated and financing was cut.
