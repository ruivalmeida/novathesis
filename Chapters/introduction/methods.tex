%!TEX root = ../../template.tex


\section{Methods and Findings}%
\label{sec:intro_methods}

The \acrlong{RQ} were used to guide the conduction of this thesis' work.
This has led to a natural two side division: on the one hand, there was
the need to algorithmically define projections for tomographic
reconstruction of a trace gas concentration field; on the other hand,
physical tests had to be conducted to ensure that the idealised
projections were feasible. These were the two hypothesis that had to be
verified.

The first hypothesis was addressed by the construction of a software
simulation system based on several projection and backprojection matrix
operators. This system is based on one of the main novelties involved in
this project: a high degree of geometric measurement freedom. This is
achieved by assuming that our tomographic acquisition system is mounted
on a custom-built \gls{UAV}, which carries specific spectroscopic
equipment and is programmed using ArduCopter's SITL software
suite~\cite{arducopter} for full autonomous operation. Spectral
acquisition takes place using a set circular trajectory that implies the
acquisition of a higher-than-usual projection number in comparison with
traditional DOAS-tomography operations. After simulating the spectral
projection information, according to a number of parameters that are
input at runtime, the system calculates their backprojection and
assembles a simulated concentration map for the selected trace gases.

The second hypothesis is addressed by means of the physical acquisition
of spectral data between two relatively close points in Almada,
Portugal. By comparing spectral information retrieved by two different
sets of equipment in said geographical points, we should be able to
establish that current day equipment is able to measure trace gas
concentrations in this kind of distances and that it is possible to make
these measurements through the difference of two passive \gls{DOAS}
measurements.
