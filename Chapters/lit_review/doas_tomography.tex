%!TEX root = ../../thesis_rui_almeida.tex
\section{DOAS Tomography}%
\label{sec:doas_tomography}





\todo[inline, color=green!60]{Editing on the mac}
\todo[inline, color=green!60!black]{Goals: describe what we found in
the DOAS tomography paper regarding the use of instrumentation,
algorithms and software.}

The first identifiable pattern is the marked prevalence of active
\gls{DOAS} systems. 11 out of the 13 retrieved papers describe or
consider an active \gls{DOAS} system of some kind. As stated in
Section~\ref{sec:doas}, active \gls{DOAS} systems do have better
analytical capabilities than their passive counterparts, although that
comes at the cost of increased instrument complexity and operational
costs. 

Another immediate conclusion is that there is a "dominant" study. Almost
half of the papers found originated from the BABII campaign, in which a
group of researchers set out to quantify pollution through \gls{DOAS}
tomography along a busy German motorway, in the beginning of the
21\textsuperscript{st} century~\cite{Pundt2005,
Laepple2004}~\todo{remaining citations here}.

All of the active \gls{DOAS} systems were purposely built for their
corresponding experiment (or group of experiments). BABII researchers
used two telescopes with around 200mm diameter and 1m focal length to
simultaneously illuminate 8 retroreflectors that were assembled onto two
towers located on each side of the road. In one of the papers associated
with this initiative, the same telescope instrumentation was used to
validate the 2D reconstruction technique that was going to be used in
the other papers.

Another important initiative with respect to \gls{DOAS} tomography was
the study conducted in 2016 by Stutz et al~\cite{Stutz2016}. The
approach in this case was to use a similar telescope to detect the light
emitted by a narrow interval UV LED light source (290nm) to create a
fence line monitoring system for Benzenes, Toluene and Xylene. The team
managed to apply this system in a successful manner in refineries in Los
Angeles and Houston. One of the most interesting aspects of this study
is that it details a tomographic system that could easily be
commercially deployed.

Another type of \gls{DOAS} tomography system was proposed by researchers
in the Cork Institute of
Technology~\cite{ODriscoll2003,ODriscoll2003a}~\todo{missing one study}.
In their three papers, the authors describe \textbf{1)} a new multipath
instrument that significantly increases the amount of projection
information in this kind of application; \textbf{2)} a tomographic
reconstruction algorithm based on evolutionary algorithms; and
\textbf{3)} the application of \gls{DOAS} tomography to a simulated
urban canyon scenario. Although all three papers present technological
innovation, it would not be fair not to say that from a strictly
literary point of view, these were among the weakest retrieved by the
search process.  

\todo[inline, color=green!60]{End editing on the mac}




