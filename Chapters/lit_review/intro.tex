%!TEX root = ../../template.tex

Chapter~\ref{cha:introduction} presents the project's background, why it
is relevant to pursue it, and a primising entry point to fulfil the
project's goals, addressing the low spatial resolution of \gls{DOAS}
through the introduction of a tomographic reconstruction process, using
the line integral nature of spectral data as projections.

Tomography is extensively used in medicine, and to some extent in the
non-destructive trials industry. It allows one to reconstruct an image
(computationally, a matrix of numeric values) from a collection of
carefully gathered projections. This subject is further discussed in
Section~\ref{sec:tomographic_algorithms_and_reconstruction_techniques}.
In Section~\ref{sec:litrev_doas_tomography}, I describe the combination
of the two disciplines (tomography and \gls{DOAS}). The section draws
heavily from the final assignment of the Advanced Software Development
course I took during the curricular part of my PhD. The assignment
consisted in writing a \gls{sms} on the subject of DOAS-tomography,
which is included in full in Appendix~\ref{ap:tomDOAS}\footnote{This
    paper was submitted for publishing in Elsevier's Remote Sensing
    Reviews journal in 2020, having been rejected, with reviewers
pointing towards several improvements. Due to lack of time, I have not
been able to pursue this further.}.  

In Section~\ref{sec:litrev_air_pollution_measurement_techniques}, I
present the major \gls{AP} monitoring techniques. While not intended to
be a thorough description of any of these techniques (each of them would
surely deserve a thesis on each own), the section aims to further
demonstrate how the system I propose is not only valid but relevant and
viable.
