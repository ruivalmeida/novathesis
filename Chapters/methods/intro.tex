%!TEX root=../../template.tex
Macroscopically, the approach to the \gls{RQ} was conducted by working
with two hypothesis:
\begin{description}
    \item[First Hypothesis:] The definition of a particular set of
        algorithmically defined projections in such a manner that they
        might be used for tomographic reconstruction of column densities
        of trace gases in the atmosphere, in a given \gls{ROI};
    \item[Second Hypothesis:] We can retrieve the column density for a
        given trace gas (or set of trace gases) between two points by
        performing a spectral measurement in both of these points in the
        same direction and subtracting them one from the other.
\end{description}

Testing the first hypothesis involved two stages:
\begin{enumerate}
    \item Idealisation, conception and design of the drone-mounted
        spectroscopic system that could perform the intended
        measurements;
    \item Development of a numeric simulation platform for the study and
        feasibility analysis of the possible measurement strategies.
\end{enumerate}

This division is also reflected throughout
Section~\ref{sec:methods_first_hypothesis}. The first few
subsections(\todo{reference subsections once they are written}) aim
towards the complete description of the designed physical systems.
Section~\ref{sub:tomosim} and onwards describe instead how the
simulation simulation framework was through and developed.

On Section~\ref{sec:the_experiment}, this chapter moves to a different
subject, which has to do with the second hypothesis. This section
describes the protocols and assumptions of an experiment designed to
test it with the equipments that were available to me at the time.

% To test the first hypothesis, I have used a number of computational
% methods to define and create projection and backprojection matrix
% operators, resulting in a dedicated simulation software tool that proves
% without a doubt that the devised projection gathering strategy is able
% to produce projection information in sufficient quantity as to perform
% tomographic reconstruction. This procedure is detailed in
% Section~\ref{sec:tomosim}. 

% The second hypothesis was experimentally tested, by the conduction of a
% number of field experiments designed to determine the validity of
% measurement hypothesis with the equipments to which I have current
% access. The experiment and the protocol that was followed is detailed in
% Section~\ref{sec:the_experiment}. This chapter is heavily based on one
% of the two articles generated by the work of this thesis.
