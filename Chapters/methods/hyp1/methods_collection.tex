%!TEX root=../../../template.tex
\subsection{Collection System}%
\label{sub:methods_collection}

The flight controller unit is in permanent communication with the
collection system, through the spectrometer controlling computer. This
is an \gls{rpi}-based single board device. This equipment controls the
spectral acquisition process, which is conducted by an Avantes Mini
spectrometer with 2048 channels and a \gls{usb} 3.0 connection. The
Avantes spectrometer can be interfaced with both Unix-based and
Microsoft Windows operating systems through a set of software libraries
that are available for download at this manufacturer's website. There
are several physical interfaces for available for both types of
operating system. Nowadays, the most common and expedient way to run
this connection is through the \gls{usb} port. This is also the
connection that allows a higher data throughput (allowing smaller
integration times to be used). In the end, the proposed system will only
run in Unix-based computers, but the Windows version was very important
for initial experiments, which at the time were programmed in C\# (code
included in Appendix~\ref{ap:c_sharp_spectrometer}). The spectrometer
control flow is very different in the two approaches. The Unix version
works by continuously polling the spectrometer for the spectral data,
which is the somewhat obvious strategy. The Windows version uses a
sophisticated by seldom used technique: Windows messages. This is a set
of fixed-value event flags that are fired at the Operating System level
and can be intercepted by running programs. Among other things, these
events allow, for instance, the detection of a new \gls{usb} connection.
Both workflows are depicted in
Figure~\ref{fig:unix_spectrometer_workflow} and
Figure~\ref{fig:windows_spectrometer_workflow}. 

\begin{figure}[htpb]
    \centering
    \missingfigure{}
    \caption{Unix type Operating System workflow for interfacing with an
    Avantes spectrometer.}%
    \label{fig:unix_spectrometer_workflow}
\end{figure}

\begin{figure}[htpb]
    \centering
    \missingfigure{}
    \caption{Windows Operating System workflow for interfacing with an
    Avantes spectrometer.}%
    \label{fig:windows_spectrometer_workflow}
\end{figure}

This type of spectrometer is usually shipped with an \gls{SMA}
connector, for direct connection of a fibre optics cable. While this is
ideal as a bench-top solution and when size and weight restrictions are
looser, we found it not to be the best fit for our particular case. We
tend to think of fibre optics as an almost lossless medium for data
transmission. This is, of course, true, but it does not hold for 




