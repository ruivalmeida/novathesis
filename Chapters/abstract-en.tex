%!TEX root = ../template.tex
%%%%%%%%%%%%%%%%%%%%%%%%%%%%%%%%%%%%%%%%%%%%%%%%%%%%%%%%%%%%%%%%%%%%
%% abstrac-en.tex
%% NOVA thesis document file
%%
%% Abstract in English([^%]*)
%%%%%%%%%%%%%%%%%%%%%%%%%%%%%%%%%%%%%%%%%%%%%%%%%%%%%%%%%%%%%%%%%%%%

\typeout{NT FILE abstrac-en.tex}

The aim of this thesis is to describe the design and development of a
proof of concept for a commercially viable large are atmospheric
analysis tool, for use in trace gas concentration mapping and
quantification. 

Atmospheric monitoring is a very well researched field, with dozens of
available analytical systems and subsystems. However, current systems
require a very important compromise between spatial and operational
complexity. We address this issue asking how we could integrate the
\gls{DOAS} atmospheric analysis technique in a \gls{UAV} with
tomographic capabilities.

Using a two-part methodology, I proposed two hypothesis for proving the
possibility of a miniaturised tomographic system, both related to how
the spectroscopic data is acquired. The first hypothesis addresses the
projection forming aspect of the acquisition, its matrix assembly and
the resolution of the consequent equations. This hypothesis was
confirmed theoretically by the development of a simulation platform for
the reconstruction of a trace gas concentration mapping.

The second hypothesis deals with the way in which data is collected in
spectroscopic terms. I proposed that with currently available equipment,
it should be possible to leverage a consequence of the Beer-Lambert law
to produce molecular density fields for trace gases using passive
\gls{DOAS}. This hypothesis was partially confirmed, with definite
conclusions being possible only through the use of complex autonomous
systems for improved accuracy.

This work has been a very important first step in the establishment of
\gls{DOAS} tomography as a commercially viable solution for atmospheric
monitoring, although further studies are required for definite results.
Moreover, this thesis has conducted to the development of a \gls{DOAS}
software library for Python that is currently being used in a production
environment. Finally, it is important to mention that two journal
articles were published from pursuing this work, both in important
journals with Impact Factors over 3.0.


% Palavras-chave do resumo em Inglês
\begin{keywords}
    \gls{DOAS}, tomography, \gls{UAV}, drones
\end{keywords} 
